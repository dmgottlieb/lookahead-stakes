\section{Model}\label{model}

Our model features two agents, Alice and Bob, attempting to meet at one
of two possible bars: the popular bar and the unpopular bar. They model
each other's decisions using nested conditioning, and choose the best
action according to the ``planning as inference'' paradigm. Each agent's
decision process begins with a prior over possible actions. All possible
outcomes are then weighted with log-probability scores according to
their utilities, and an action is sampled from the posterior with a
softmax process. Each agent's decision is then characterized by the
following probability distribution:

\[P(\text{action}) \propto e^{\alpha EU(\text{action})},\]

where \(EU(\text{action})\) is the expected utility of the action and
\(\alpha\) is a softmax temperature parameter (low \(\alpha\) = high
temperature).

In the model, the main possible actions are
\texttt{\{\textquotesingle{}popular-bar\textquotesingle{},\ \textquotesingle{}unpopular-bar\textquotesingle{}\}}.
Both agents have the same prior over actions, which is a 55\% chance of
choosing the popular bar (this bias reflects that the popular bar is the
more \emph{salient} option). They then have utility payoffs assigned to
game outcomes. For example, in the baseline symmetric game, each agent
receives a payoff of 1 if they coordinate on the same bar, and 0
otherwise. Finally, they also have an implicit resource cost to
recursion, encoded by a \texttt{rate} parameter. The \texttt{rate}
parameter reflects the chance, at the \(n\)th level of recursion, to
descend to level \(n+1\).

The core of the model is recursive nested conditioning: each agent
decides what to do by simulating the other agent, and assigning larger
utility weights to outcomes in which the agent and the simulated
opponent choose the same action. For example, here is Alice's core loop:

\begin{verbatim}
var alice = function(depth,rate, alpha) {
  return Infer(opts, function(){
    var myLocation = locationPrior();
    if ((depth === 0) || (!flip(rate))) {
      return myLocation;
    } else {
      var bobLocation = sample(bob(depth - 1,rate, alpha));
      var payoff = utility('alice', myLocation, bobLocation);
      factor(alpha*(payoff));
      return myLocation;
    }
  });
};
\end{verbatim}

\texttt{depth} is the maximum possible recursion depth -- in the
prototype, we usually set this parameter to 5 for computational
tractability. \texttt{rate} is the prior over the decision, at each
level of recursion, to go one level deeper. If \texttt{rate} is 0.5
then, \emph{ceteris paribus}, the agent will do at least one level of
recursion 0.5 of the time, at least two levels of recursion
\((0.5)^2 = 0.25\) of the time, etc. This reflects our novel modeling
assumption, that deeper recursions are more expensive. In the case in
which agents are indifferent among all outcomes, then we can derive the
cost of each level of recursion as follows:

\begin{align*} 
P(n\text{ levels}) &\propto e^{EU(n\text{ levels})},\\
P(n\text{ levels}) &= \frac{1}{2^{n+1}} \\
                   &= e^{-(\ln 2)(n+1)}
\end{align*}

In other words, \texttt{rate} \(> 0\) corresponds to a utility cost
that's linear on recursion depth, here expressed as \(\ln 2\) per
recursion level.

\texttt{alpha} is the softmax temperature as discussed above.

\section{Results}\label{results}

Here we report preliminary results from applying the model in several
different conditions:

\begin{enumerate}
\def\labelenumi{\arabic{enumi}.}
\itemsep1pt\parskip0pt\parsep0pt
\item
  Baseline (recursion is free).
\item
  Bounded (recursion is costly).
\end{enumerate}

\begin{enumerate}
\def\labelenumi{\alph{enumi}.}
\itemsep1pt\parskip0pt\parsep0pt
\item
  Low stakes.
\item
  High stakes.
\end{enumerate}

\begin{enumerate}
\def\labelenumi{\arabic{enumi}.}
\setcounter{enumi}{2}
\itemsep1pt\parskip0pt\parsep0pt
\item
  Asymmetric (agents prefer different bars).
\end{enumerate}

\begin{enumerate}
\def\labelenumi{\alph{enumi}.}
\itemsep1pt\parskip0pt\parsep0pt
\item
  Low stakes.
\item
  High stakes.
\end{enumerate}

\begin{enumerate}
\def\labelenumi{\arabic{enumi}.}
\setcounter{enumi}{3}
\itemsep1pt\parskip0pt\parsep0pt
\item
  Asymmetric (agents prefer the same bar to different degrees).
\end{enumerate}

\begin{enumerate}
\def\labelenumi{\alph{enumi}.}
\itemsep1pt\parskip0pt\parsep0pt
\item
  Low stakes.
\item
  High stakes.
\end{enumerate}

\subsection{Baseline}\label{baseline}

To make recursion cost-free, we set the \texttt{rate} parameter to 1.0.
Agents will then always use the maximal allowed recursion depth (in this
case 5).

In this condition, agents successfully coordinate about 64\% of the
time. In particular, they coordinate on the popular bar 59\% of the
time, and the unpopular bar 5\% of the time.

\subsection{Bounded}\label{bounded}

Here, recursion has a resource cost that is linear on recursion depth,
as discussed above. Setting \texttt{rate} to 0.5 results in a resource
cost of about 0.69 in the utility units of the payoff.

In the low stakes condition, the payoffs are (1,1) for successfully
coordinating and (0,0) for failing. So we expect the costs of recursion
to quickly outrun the expected benefits. The results reflect this. In
this condition, the agents successfully coordinate only about 51\% of
the time, 33\% at the popular bar and 18\% at the unpopular bar.

In the high stakes condition, we set the payoffs to (3,3) for successful
coordination (this increase is comparable to the high stakes
differential in experimental papers). Now, the rewards for success
should be large enough to offset the costs of increased recursion depth.
Again, the results reflect this. In this condition, agents successfully
coordinate about 57\% of the time, 47\% at the popular bar and 10\% at
the unpopular bar.

\subsection{Asymmetric (agents prefer different
bars)}\label{asymmetric-agents-prefer-different-bars}

The experimental literature has found that this kind of payoff asymmetry
destroys coordination. Our model correctly predicts this finding.

In the low stakes condition, we set the payoffs to (1.2,1) and (1,1.2)
respectively for the two agents, reflecting that both want to coordinate
but they prefer to coordinate on different bars. In this condition, they
coordinate about 50\% of the time, 33\% at the popular bar and 17\% at
the unpopular bar.

In the high stakes condition, we set the payoffs to (3.6,3) and (3,3.6)
respectively for the two agents. This increase is comparable to the
differential in the high stakes condition in experiments that have
tested behavior in asymmetric high stakes conditions. In this condition,
the agents coordinate about 55\% of the time, 47\% on the popular bar
and 8\% on the unpopular bar. This seems somewhat at odds with the
experimental results. The probable explanation is that the higher stakes
enabled the agents to reason to a higher recursive depth, which
amplified the effect of the salience bias towards the popular bar (note
that the agents are also much more likely to go to the popular bar).

More modeling and analysis is required.

\subsection{Asymmetric (agents prefer the same bar to different
degrees)}\label{asymmetric-agents-prefer-the-same-bar-to-different-degrees}

{[}To come.{]}

\section{Further work}\label{further-work}

We are largely done with the conceptual work of the project. Several
engineering points remain outstanding.

In particular:

\begin{itemize}
\item
  We are not sure the current model handles the payoffs correctly.

  In the code block above, you can see that
  \texttt{factor(alpha*(payoff))} is added at each level of recursive
  simulation. This seems to have the effect that the payoff for
  coordinating on the same bar can be double-counted, or in fact counted
  up to \(d\) times where \(d\) is the depth parameter. Unfortunately,
  this seems to be worse than simply being off by a constant factor,
  because it means deep recursions get a \texttt{factor} bonus that
  increases as they go deeper. The coordination payoff is supposed to
  offset the resource cost of recursion, but the offset should be a flat
  amount, not proportional to the depth.

  If this diagnosis is correct, we're not sure what the correct
  theoretically motivated fix of this problem should be.
\item
  We would like to report a richer range of results for each of our
  trials. Currently, we report only the joint distribution over which
  bars the agents go to, which allows us to calculate the rate of
  successful coordination. However, we would also like to report the
  distribution of actual recursion depth induced by a particular setting
  of \texttt{rate} and utilities. This would allow us to examine in a
  more fine-grained way (1) how stakes influence recursion depth, and
  (2) how much increased recursion depth helps achieve good outcomes in
  various conditions.
\end{itemize}
